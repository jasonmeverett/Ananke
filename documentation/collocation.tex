\subsection{Collocation Overview}
The decision vector for this type of multi-phase collocation is large, but not untraceable. For sake of clarity, a P-phase collocation method example will be shown with n collocation points in each phase (note: the amount of collocation points can vary per phase). It is also important to note that this matrix can be rearranged in any fashion, as some forms may bode well for NLP solving schemes that require specific gradient formats.
\begin{multicols}{2}
\begin{equation}
\vec{X}_{dv} = 
\begin{bmatrix}
\vec{R}_{[x ... z]_{[0 ... P]}} \\
\vec{V}_{[x ... z]_{[0 ... P]}} \\
\vec{U}_{[x ... z]_{[0 ... P]}} \\
\vec{m}_{[0 ... P]} \\
\vec{\eta }_{[0 ... P]} \\
\nu_0 \\
T_{[0 ... P]} \\
\end{bmatrix}
\rightarrow
\begin{bmatrix}
3 n_0 + ... + 3 n_p \\
3 n_0 + ... + 3 n_p \\
3 n_0 + ... + 3 n_p \\
n_0 + ... + n_p \\
n_0 + ... + n_p \\
1 \\
P 
\end{bmatrix}
\end{equation} 
\break
\begin{equation}
\vec{F} = 
\begin{bmatrix}
J \\
\vec{R}_0 \\
\vec{V}_0 \\
m_{0_{[0 ... P]}} \\
\vec{R}_f \\
\vec{V}_f \\
\vec{R}_{m_{[0 \rightarrow P]}} \\
\vec{V}_{m_{[0 \rightarrow P]}} \\
\vec{R}_{c_{[0 ... P]}} \\
\vec{V}_{c_{[0 ... P]}} \\
\vec{m}_{c_{[0 ... P]}} \\
\vec{U}_{mag_{[0 ... P]}} \\
m_{f_{[0 ... P]}} \\
\vec{\eta}_{lb_{[0 ... P]}} \\
\vec{\eta}_{ub_{[0 ... P]}} \\
\nu_{lb_{0}} \\
\nu_{ub_{0}} \\
T_{lb_{[0 ... P]}} \\
T_{ub_{[0 ... P]}}  
\end{bmatrix}
\rightarrow
\begin{bmatrix}
1 \\
3 \\
3 \\
P? \\
3 \\
3 \\
3 (P - 1) \\
3 (P - 1) \\
3(n_0-1) + ... + 3(n_P - 1) \\
3(n_0-1) + ... + 3(n_P - 1) \\
(n_0-1) + ... + (n_P - 1) \\
n_0 + ... + n_p \\
P \\
n_0 + ... + n_p \\
n_0 + ... + n_p \\
1 \\
1 \\
P \\
P \\
\end{bmatrix}
\end{equation} 
\end{multicols}
Assume an example of a 3-phase trajectory with each phase containing a total of 15 collocation points. This would lead to a decision vector with a length of --- and a fitness/constraint vector of length  ---. The constraint vector components will be described throughout the subsequent sections. The true form of the constraints may vary at mechanization depending on the NLP solver selected.


\subsection{Path Constraints}
Under trapezoidal collocation transcription, the path constraints of $N$ collocation points will take on the following form within any arbitrary phase of the trajectory:
\begin{align}
\vec{R}_{k+1} - \vec{R}_k = 
\frac{1}{2}\Delta t ( \vec{V}_{k+1} + \vec{V}_k ) &&
k \in \{ 1, ... , N \}
\end{align}
\begin{align}
\vec{V}_{k+1} - \vec{V}_k = 
\frac{1}{2}\Delta t \left[ \frac{\mu }{R_{k+1}^3}\vec{R}_{k+1} + \frac{\mu }{R_{k}^3}\vec{R}_{k} +  \frac{T_m \eta_k}{m_k} \hat{U}_{k} + \frac{T_m \eta_{k+1}}{m_{k+1}} \hat{U}_{k+1} \right] &&
k \in \{ 1, ... , N \}
\end{align}
\begin{align}
m_{k} - m_{k+1} = 
\frac{1}{2}\Delta t \frac{T_m}{g_0 I_{sp}} ( \eta_{k+1} + \eta_{k} ) &&
k \in \{ 1, ... , N \}
\end{align}
\subsection{Boundary Constraints}
\subsubsection{Initial and Final Trajectory States}
\paragraph{}
It is required in this nonlinear program that the final targeted position and velocity remain constant in the planet-fixed frame. This requirement transcribes itself into a nonlinear equality constraint on the last collocation point of the last phase of the trajectory. In phase $P$ of $P$ with $N$ collocation points:
\begin{equation}
\vec{R}_{des_{f}}^{I} = \vec{R}_{t/c}^{I} = C_{\mathit{UEN}}^{I} \vec{R}_{t/c}^{\mathit{UEN}}
\end{equation}
\begin{equation}
\vec{V}_{des_{f}}^{I} = \vec{V}_{t/c}^{I} + \vec{\Omega} \times \vec{R}_{t/c}^{I} = C_{\mathit{UEN}}^{I} \vec{V}_{t/c}^{\mathit{UEN}} + \vec{\Omega} \times \vec{R}_{t/c}^{I}
\end{equation}
Where  $\vec{R}_{t/c}^{\mathit{UEN}}$ represents the target position vector expressed in the $\mathit{UEN}$ frame (which is a fixed offset from the $\mathit{LS}$ frame).
\paragraph{}
Also, for this problem, it is assumed that the initial orbit of the vehicle is fixed - that is, of the 6 necessary parameters to fully constrain an orbital state expressed in Kelperian elements, only the initial true anomaly of the orbit is allowed to be optimized. This allows for simpler derivation of partials, but requires that the vehicle's initial state always begin in a reference orbital plane. Choosing true anomaly $\nu$ as a decision vector parameter is the simplest way to approach this for both constraint and partial development. Given a certain right-ascension of the ascending node $\Omega$, inclination $i$, eccentricity $e$, semi-major axis $a$ and argument of periapsis $\omega$, the starting position and velocity constraint can be expressed as a function of $\nu$ through the following equations:
\begin{equation}\label{eq:state0pos}
\vec{R}_{des_0} = R_z (-\Omega) R_x(-i) R_z(-\omega) \vec{o}
\end{equation}
\begin{equation}\label{eq:state0vel}
\vec{R}_{des_0} = R_z (-\Omega) R_x(-i) R_z(-\omega) \dot{\vec{o}}
\end{equation}
Where:
\begin{equation}
E = 2 \arctan{\left(\sqrt{\frac{1-e}{1+e}} \tan{\frac{\nu}{2}}\right)}
\end{equation}
\begin{equation}
r_c = a(1 - e \cos{E})
\end{equation}
\begin{align}
\vec{o} = r_c \begin{bmatrix}
\cos {\nu} \\ \sin {\nu} \\ 0
\end{bmatrix} &&
\dot{\vec{o}} = \frac{\sqrt{\mu a}}{r_c} 
\begin{bmatrix}
-\sin {E} \\ \sqrt{1 - e^2} \cos{E} \\ 0
\end{bmatrix}
\end{align}

\subsubsection{Phase Boundary Constraints}
At the boundaries of each phase, a constraint is set on the state of the vehicle. In other words, for phase $i$ with $n$ segments and phase $i+1$ as the subsequent phase, the following constraints must be met:
\begin{equation}
\vec{R}_{{i}_N} = \vec{R}_{{i+1}_1}
\end{equation}
\begin{equation}
\vec{V}_{{i}_N} = \vec{V}_{{i+1}_1}
\end{equation}
For mass, there are two forms of constraints that can be employed. The first is representative of mass continuity, when a phase break does not represent a staging of the vehicle:
\begin{equation}
m_{{i}_N} = m_{{i+1}_1} + \Delta m_{i \rightarrow i+1}
\end{equation}
where $\Delta m_{i \rightarrow i+1}$ is a \textit{fixed} mass drop that is allowed to occur from phase $i$ to phase $i+1$. This phase delta could provide use if a fixed-mass component of the vehicle must be jettisoned at a phase boundary. Another form that is more synonymous to a staging of a vehicle requires an equality constraint on the initial mass of phase $i+1$:
\begin{equation}
mp_{{i+1}_1} = mp_{\mathit{init}_{i+1}}
\end{equation}
Lastly, it is clear that an inequality constraint is required to ensure the vehicle does not spend more propellant in a specific stage than it actually has. In phase $i$ with $N$ collocation points, this represents itself as:
\begin{equation}
mp_{i_{N}} >= 0
\end{equation}
\subsubsection{Other Constraints}
An altitude constraint is optional, because in most cases, the most optimal trajectory for a descent is not a trajectory that intersects the planet before necessary. However, one can impose an altitude restriction on certain portions of flight using the following equation:
\begin{equation}
R_x^2 + R_y^2 + R_z^2 - R_{min}^2 >= 0
\end{equation}
Where $R_{min}$ can vary each phase. This constraint is not covered in this derivation. Also required are throttling constraints:
\begin{equation}
\eta_{lb} <= \eta_{k}
\end{equation}
\newpage
\subsection{Gradient}
The gradient is practically always the most desirable key to a nonlinear program that usually speeds up a solver by orders of magnitude. Luckily, for the problem described in the sections above, the partials can be solved analytically with a healthy amount of dedication and patience. The gradient for this problem, i.e. the partial of the fitness vector (the objective function and all constraints) with respect to the decision vector, is outlined below.

\begin{equation}
\pdv{\vec{X}}{\vec{F}} = 
\begin{bmatrix}

[0] &
[0] &
[0] &
\pdv{J}{\vec{m}_{[1 ... P]}} &
\pdv{J}{\vec{\eta}_{[1 ... P]}} &
[0]  &
\pdv{J}{T_{[1 ... P]}} \\


\pdv{\vec{R}_0}{\vec{R}_{[xyz]_{[1 ... P]}}} &
[0] &
[0] &
[0] &
[0] &
\pdv{\vec{R}_0}{\nu_0}  &
[0] \\

[0] &
\pdv{\vec{V}_0}{\vec{V}_{[xyz]_{[1 ... P]}}} &
[0] &
[0] &
[0] &
\pdv{\vec{V}_0}{\nu_0}  &
[0] \\

[0] &
[0] &
[0] &
\pdv{m_{0_{[1 ... P]}}}{\vec{m}_{[1 ... P]}} &
[0] &
[0] &
[0]  \\

\pdv{\vec{R}_f}{\vec{R}_{[xyz]_{[1 ... P]}}} &
[0] &
[0] &
[0] &
[0] &
[0] &
\pdv{\vec{R}_f}{T_{[1 ... P]}} \\

[0] &
\pdv{\vec{V}_f}{\vec{V}_{[xyz]_{[1 ... P]}}} &
[0] &
[0] &
[0] &
[0] &
\pdv{\vec{V}_f}{T_{[1 ... P]}} \\

\pdv{\vec{R}_{m_{[1 \rightarrow 2 ... P-1 \rightarrow P]}}}{\vec{R}_{[xyz]_{[1 ... P]}}} &
[0] &
[0] &
[0] &
[0] &
[0] &
[0]  \\

[0] &
\pdv{\vec{V}_{m_{[1 \rightarrow 2 ... P-1 \rightarrow P]}}}{\vec{V}_{[xyz]_{[1 ... P]}}} &
[0] &
[0] &
[0] &
[0] &
[0]  \\

\pdv{\vec{R}_{c_{[xyz]_{[1 ... P]}}}}{\vec{R}_{[xyz]_{[1 ... P]}}} &
\pdv{\vec{R}_{c_{[xyz]_{[1 ... P]}}}}{\vec{V}_{[xyz]_{[1 ... P]}}} &
[0] &
[0] &
[0] &
[0] &
\pdv{\vec{R}_{c_{[xyz]_{[1 ... P]}}}}{T_{[1 ... P]}} \\

\pdv{\vec{V}_{c_{[xyz]_{[1 ... P]}}}}{\vec{R}_{[xyz]_{[1 ... P]}}} &
\pdv{\vec{V}_{c_{[xyz]_{[1 ... P]}}}}{\vec{V}_{[xyz]_{[1 ... P]}}} &
\pdv{\vec{V}_{c_{[xyz]_{[1 ... P]}}}}{\vec{U}_{[xyz]_{[1 ... P]}}} &
\pdv{\vec{V}_{c_{[xyz]_{[1 ... P]}}}}{\vec{m}_{[1 ... P]}} &
\pdv{\vec{V}_{c_{[xyz]_{[1 ... P]}}}}{\vec{\eta}_{[1 ... P]}} &
[0] &
\pdv{\vec{V}_{c_{[xyz]_{[1 ... P]}}}}{T_{[1 ... P]}} \\

[0] &
[0] &
[0] &
\pdv{\vec{m}_{c_{[1 ... P]}}}{\vec{m}_{[1 ... P]}} &
\pdv{\vec{m}_{c_{[1 ... P]}}}{\vec{\eta}_{[1 ... P]}} &
[0] &
\pdv{\vec{m}_{c_{[1 ... P]}}}{T_{[1 ... P]}} \\ 

[0] &
[0] &
\pdv{\vec{U}_{mag_{[1 ... P]}}}{\vec{U}_{[xyz]_{[1 ... P]}}} &
[0] &
[0] &
[0] &
[0]  \\

[0] &
[0] &
[0] &
\pdv{m_{f_{[1 ... P]}}}{\vec{m}_{[1 ... P]}} &
[0] &
[0] &
[0]  \\

[0] &
[0] &
[0] &
[0] &
\pdv{\vec{\eta}_{lb_{[1 ... P]}}}{\vec{\eta}_{[1 ... P]}} &
[0] &
[0]  \\

[0] &
[0] &
[0] &
[0] &
\pdv{\vec{\eta}_{ub_{[1 ... P]}}}{\vec{\eta}_{[1 ... P]}} &
[0] &
[0]  \\

[0] &
[0] &
[0] &
[0] &
[0] &
\pdv{T_{lb_{[1 ... P]}}}{T_{[1 ... P]}} &
[0]  \\

[0] &
[0] &
[0] &
[0] &
[0] &
\pdv{T_{ub_{[1 ... P]}}}{T_{[1 ... P]}} &
[0]  \\

[0] &
[0] &
[0] &
[0] &
[0] &
[0] &
\pdv{\nu_{lb_0}}{\nu_0} \\

[0] &
[0] &
[0] &
[0] &
[0] &
[0] &
\pdv{\nu_{ub_0}}{\nu_0} \\

\end{bmatrix}
\end{equation}
The equation subscripts in the next section often take on a standardized form for readibility. For example, take the following partial:
\begin{equation}
\pdv{R_{c_{X_{1_{1}}}}}{R_{X_{1_{2}}}}
\end{equation}
This represents partial of the path constraint for $R_X$ for phase $1$ and for collocation point $1$, with respect to the $R_X$ collocation point $2$ in phase $1$. Also, when not specified, the term $\Delta t$ for a specific phase $i$ represents:
\begin{equation}
\Delta t_i = T_i / N_i
\end{equation}
Also, when not specified, for phase $i$, $T_m = T_{m_i}$ and $n = n_i$.

\newpage
\subsubsection{Objective Function - Minimum Control}
This is for the minimum control problem of the entire trajectory for all phases with equal weighting, i.e.:
\begin{equation}
J = \sum_{i=1}^{P} \sum_{k=1}^{n_i-1} \frac{1}{2} \Delta t_i
\left[
\left(
\frac{T_{m_i} \eta_{i_k}}{m_{i_k}}
\right)^2 + 
\left(
\frac{T_{m_i} \eta_{i_{k+1}}}{m_{i_{k+1}}}
\right)^2
\right]
\end{equation}
Calculating required partials:

\begin{equation}
\pdv{J}{\vec{m}_{[1 ... P]}}  =
\underset{(1) \times (n_1 + ... + n_P)}{
\begin{bmatrix}
\pdv{J}{\vec{m}_1} &
\pdv{J}{\vec{m}_2} & 
\dots &
\pdv{J}{\vec{m}_P}
\end{bmatrix}
}
\end{equation}

\begin{equation}
\pdv{J}{\vec{m}_{1}} = 
\underset{(1) \times (n_1) }{
\begin{bmatrix}
-\Delta t_1 \frac{T_{m_1}^2 \eta_{1_1}^2 }{m_{1_1}^3} & 
-2 \Delta t_1 \frac{T_{m_1}^2 \eta_{1_2}^2 }{m_{1_2}^3} & 
... & 
-2 \Delta t_1 \frac{T_{m_1}^2 \eta_{1_{n-1}}^2 }{m_{1_{n-1}}^3} & 
-\Delta t_1 \frac{T_{m_1}^2 \eta_{1_{n}}^2 }{m_{1_{n}}^3} & 
\end{bmatrix}
}
\end{equation}
This pattern is repeated for $\pdv{J}{\vec{m}_2} ... \pdv{J}{\vec{m}_P}$.

\begin{equation}
\pdv{J}{\vec{\eta}_{[1 ... P]}}  =
\underset{(1) \times (n_1 + ... + n_P)}{
\begin{bmatrix}
\pdv{J}{\vec{\eta}_1} &
\pdv{J}{\vec{\eta}_2} & 
\dots &
\pdv{J}{\vec{\eta}_P}
\end{bmatrix}
}
\end{equation}

\begin{equation}
\pdv{J}{\vec{\eta}_{1}} = 
\underset{(1) \times (n_1) }{
\begin{bmatrix}
\Delta t_1 \frac{T_{m_1}^2 \eta_{1_1} }{m_{1_1}^2} & 
2 \Delta t_1 \frac{T_{m_1}^2 \eta_{1_2} }{m_{1_2}^2} & 
... & 
2 \Delta t_1 \frac{T_{m_1}^2 \eta_{1_{n-1}} }{m_{1_{n_-1}}^2} & 
\Delta t_1 \frac{T_{m_1}^2 \eta_{1_{n}} }{m_{1_{n}}^2} & 
\end{bmatrix}
}
\end{equation}
This pattern is repeated for $\pdv{J}{\vec{\eta}_2} ... \pdv{J}{\vec{\eta}_P}$.

\begin{equation}
\pdv{J}{T_{[1 ... P]}} = 
\underset{(1) \times (P)}{
\begin{bmatrix}
\frac{J_1}{T_1} &
\frac{J_2}{T_2} &
\dots &
\frac{J_n}{T_n}
\end{bmatrix}
}
\end{equation}
Where
\begin{equation}
J_1 = \left\{
\sum_{k=1}^{n_i-1} \frac{1}{2} \Delta t_i   
\left[
\left(
\frac{T_{m_i} \eta_{i_k}}{m_{i_k}}
\right)^2 + 
\left(
\frac{T_{m_i} \eta_{i_{k+1}}}{m_{i_{k+1}}}
\right)^2
\right]
\right\}_{i=1}
\end{equation}

\newpage
\subsubsection{Initial Position Equality Constraint}
The constraint takes on the form:
\begin{equation}
\vec{R}_0 = \vec{R}_{1_1} - \vec{R}_{des_0}
\end{equation}
\begin{equation}
\pdv{\vec{R}_0}{\vec{R}_{[XYZ]_{[1 ... P]}}} = 
\underset{(3) \times (3 n_1 + ... + 3 n_P)}{ 
\begin{bmatrix}
\pdv{\vec{R}_0}{\vec{R}_{X_{[1 ... P]}}} & 
\pdv{\vec{R}_0}{\vec{R}_{Y_{[1 ... P]}}} & 
\pdv{\vec{R}_0}{\vec{R}_{Z_{[1 ... P]}}} 
\end{bmatrix} }
\end{equation}

\begin{equation}
\pdv{\vec{R}_0}{\vec{R}_{X_{[1 ... P]}}} = 
\underset{(3) \times (n_1 + ... + n_p)}{
\begin{bmatrix}
\pdv{\vec{R}_0}{\vec{R}_{X_1}} &
\pdv{\vec{R}_0}{\vec{R}_{X_2}} &
\dots &
\pdv{\vec{R}_0}{\vec{R}_{X_P}} &
\end{bmatrix}
} = 
\begin{bmatrix}
\underset{(3) \times (n_1)}{
\begin{bmatrix}
1 & 0 & \dots & 0 \\
0 & 0 & \dots & 0 \\
0 & 0 & \dots & 0
\end{bmatrix}
} & 
\underset{(3) \times (n_2)}{
\begin{bmatrix}
0
\end{bmatrix}
} &
\dots &
\underset{(3) \times (n_P)}{
\begin{bmatrix}
0
\end{bmatrix}
}
\end{bmatrix}
\end{equation}

Similarly,

\begin{equation}
\pdv{\vec{R}_0}{\vec{R}_{Y_{[1 ... P]}}} = 
\begin{bmatrix}
\underset{(3) \times (n_1)}{
\begin{bmatrix}
0 & 0 & \dots & 0 \\
1 & 0 & \dots & 0 \\
0 & 0 & \dots & 0
\end{bmatrix}
} & 
\underset{(3) \times (n_2)}{
\begin{bmatrix}
0
\end{bmatrix}
} &
\dots &
\underset{(3) \times (n_P)}{
\begin{bmatrix}
0
\end{bmatrix}
}
\end{bmatrix}
\end{equation}

\begin{equation}
\pdv{\vec{R}_0}{\vec{R}_{Z_{[1 ... P]}}} = 
\begin{bmatrix}
\underset{(3) \times (n_1)}{
\begin{bmatrix}
0 & 0 & \dots & 0 \\
0 & 0 & \dots & 0 \\
1 & 0 & \dots & 0
\end{bmatrix}
} & 
\underset{(3) \times (n_2)}{
\begin{bmatrix}
0
\end{bmatrix}
} &
\dots &
\underset{(3) \times (n_P)}{
\begin{bmatrix}
0
\end{bmatrix}
}
\end{bmatrix}
\end{equation}

\begin{equation}
\pdv{\vec{R}_0}{\nu_0} = -\pdv{\vec{R}_{des_0}}{\nu_0} = 
- R_z (-\Omega) R_x(-i) R_z(-\omega) \pdv{}{\nu}\left[ \vec{o} \right]
\end{equation}

\begin{equation}
\pdv{}{\nu}\left[ \vec{o} \right] = 
\pdv{}{\nu}\left\{ r_c
\begin{bmatrix}
\cos {\nu} \\ \sin {\nu} \\ 0
\end{bmatrix}  \right\} = 
\pdv{r_c}{\nu}
\begin{bmatrix}
\cos {\nu} \\ \sin {\nu} \\ 0
\end{bmatrix} + 
r_c
\begin{bmatrix}
-\sin {\nu} \\ \cos {\nu} \\ 0
\end{bmatrix} 
\end{equation}

\begin{equation}
\pdv{r_c}{\nu} = 
e \sin{(E)} \pdv{E}{\nu}
\end{equation}

\begin{align}
\pdv{E}{\nu} = \frac{2}{1 + B^2} \pdv{B}{\nu} &&
B = \sqrt{\frac{1-e}{1+e}} \tan{\frac{\nu}{2}}
\end{align}
\begin{align}
\pdv{B}{\nu} = \frac{1}{2}  \sqrt{\frac{1-e}{1+e}}  \sec^2{\frac{\nu}{2}}
\end{align}

\newpage
\subsubsection{Initial Velocity Constraint}
The constraint takes on the form:
\begin{equation}
\vec{V}_0 = \vec{V}_{1_1} - \vec{V}_{des_0}
\end{equation}
\begin{equation}
\pdv{\vec{V}_0}{\vec{V}_{[xyz]_{[1 ... P]}}} = \pdv{\vec{R}_0}{\vec{R}_{[XYZ]_{[1 ... P]}}}
\end{equation}

\begin{equation}
\pdv{\vec{V}_0}{\nu_0} = -\pdv{\vec{V}_{des_0}}{\nu_0} = 
- R_z (-\Omega) R_x(-i) R_z(-\omega) \pdv{}{\nu}\left[ \dot{\vec{o}} \right]
\end{equation}

\begin{equation}
\dv{}{\nu}\left[ \dot{\vec{o}} \right] = \sqrt{\mu a}  \dv{}{\nu} \left\{  
\begin{bmatrix}
\frac{-\sin {E}}{r_c} \\ \frac{\sqrt{1 - e^2} \cos{E}}{r_c} \\ 0
\end{bmatrix}
\right\}
\end{equation}

\subsubsection{Initial Mass Constraints}
For the special case of initialization mass:
\begin{equation}
m_{0_1} = m_{1_1} - m_{tot}
\end{equation}
\begin{equation}
\pdv{m_{0_1}}{\vec{m}_{[1 ... P]}}
\end{equation}



